\documentclass[11pt]{article}
\usepackage{tabularx}
\usepackage{booktabs}
\usepackage[left=2cm, right=2cm]{geometry}
\begin{document}
	\begin{titlepage}
	\begin{center}
		\vspace*{1cm}
		\LARGE
		\textbf{Organic Chemistry 2 Lab Report}
		
		\vspace{0.5cm}
		\Large
		Chemical Identification
		
		\vspace{1.5cm}
		
		\textbf{Cameron Good}
		
		\vfill
		\large
		Report for Eastern Iowa Community Colleges \\
		Organic Chemistry 2
		2024\\
		
		
		\vspace{0.5cm}
		Chemistry\\
		Eastern Iowa Community Colleges\\
		United States\\
		05/01/2024
		
	\end{center}
\end{titlepage}
	\thispagestyle{plain}
\begin{center}
	\Large
	\textbf{Organic Chemistry 2 Lab Report}
	
	\vspace{0.4cm}
	\large
	Chemical Identification
	
	\vspace{0.4cm}
	\textbf{Cameron Good}
	
	\vspace{0.9cm}
	\textbf{Introduction}
\end{center}
\paragraph{}
The purpose of this laboratory is to identify four chemical unknowns through determination of physical properties of the unknowns and their derivatives, chemical reactivity, and spectroscopic methods. The unknowns have a primary functional group of an Aldehyde, an Amine, a Ketone, an Alcohol, a Carboxylic Acid, an Ester, or a Phenol group. They may have also have an additional secondary or tertiary functional group of a halogen, a nitro, a cyano, an alkoxy, an alkene, an alkyne, or an aromatic group.

Chemical reactivity can be used to identify substituents on a carbon chain. The ability of a substance to dissolve can used to identify whether a molecule is polar or non-polar or if it is an acid or base. Numerous functional groups can be detected using with qualitative tests. Physical properties like melting points and boiling points can be used to look up chemicals in a reference book that may be potential matches. Finally NMR, IR, and mass spectrometry can be used to estimate the structure and mass of the molecule with a high degree of accuracy.


	\pagebreak
	\section{Data Tables}
	\subsection{Unknown C1}
	\begin{table}[h]
		\raggedright
		\begin{tabular}{ l  p{3.0cm} p{5.4cm} } 
			\toprule
			\textbf{Test} & \textbf{Result} & \textbf{Conclusion} \\\midrule
			
			$H_2O$ Solubility & Not Soluble & C1 is a relatively non-polar molecule \\\hline
			
			$NaOH$ Solubility & Not Soluble & C1 is not an acid \\\hline
			
			$HCl$ Solubility & Not Soluble & C1 is not a base  \\\hline
			
			$H_2SO_4$ Solubility & Reactive & C1 is neutral \\\hline
			
			Beilstein Test & No green flame, Negative & C1 does not have a halide substituent \\\hline
			
			Soot Formation & No soot & C1 is unlikely to be aromatic or have an alkene group \\\hline
			
			$H_2CrO_4$ & Positive Blue/Green Precipitate & C1 is likely an alcohol or aldehyde \\\hline
			
			Bromine & No reaction & C1 is unlikely to have an alkene group and is likely not an aldehyde \\\hline 
			
			2,4-DNPH & No reaction & C1 is not an aldehyde or ketone \\\hline 
			
			$KMnO_4$ & No reaction & C1 is not an alkene \\\hline  
			
			Ferrous Hydroxide & No reaction & No nitro group \\\hline
			
			$AgNO_3$ Test & No reaction & C1 does not have a tertiary alkyl halide moiety \\\hline 
			
			$NaI$ Test & No reaction & C1 does not have a primary alkyl halide moiety \\\hline 
			
			C1 Boiling Point & 141 C &  \\\hline
			
			3,5-Dinitrobenzoate Deriv. MP & 92 C &  \\\hline 
			
			Phenylurethane Deriv. MP & 127 C &  \\\hline 
			
			
			
		\end{tabular}
	\end{table}
	\pagebreak
	
	\subsection{Unknown C2}
	\begin{table}[h]
		\raggedright
		\begin{tabular}{ l  p{3.0cm} p{5.4cm} } 
			\toprule
			\textbf{Test} & \textbf{Result} & \textbf{Conclusion} \\\midrule
			
			$H_2O$ Solubility & Not Soluble & C2 is a relatively non-polar molecule \\\hline
			
			$NaOH$ Solubility & Not Soluble & C2 is not an acid \\\hline
			
			$HCl$ Solubility & Not Soluble & C2 is not a base  \\\hline
			
			$H_2SO_4$ Solubility & Reactive & C2 is neutral \\\hline
			
			Beilstein Test & No green flame, Negative & C2 does not have a halide substituent \\\hline
			
			Soot Formation & No soot & C2 is unlikely to be aromatic or have an alkene group \\\hline
			
			$H_2CrO_4$ & Produced orange liquid, Negative & C2 is not a primary or secondary alcohol nor is it an aldehyde \\\hline
			
			Bromine & No reaction & C2 is unlikely to have an alkene or aldehyde group \\\hline 
			
			2,4-DNPH & Yellow/Orange Precipitate & C2 has a carbonyl group \\\hline 
			
			$KMnO_4$ & No reaction & C2 is not an alkene \\\hline 
			
			Ferrous Hydroxide & No reaction & No nitro group \\\hline 
			
			$AgNO_3$ Test & No reaction & C2 does not have a tertiary alkyl halide moiety \\\hline 
			
			$NaI$ Test & No reaction & C2 does not have a primary alkyl halide moiety \\\hline 
			
			C2 Boiling Point & 105 C &  \\\hline
			
			2,4-DNPH Deriv. MP & 93 C &  \\\hline
			
			Semicarbazone Deriv. MP & 236 C &  \\\hline
			
			
			
		\end{tabular}
	\end{table}
	\pagebreak
	\subsection{Unknown C3}
	\begin{table}[h]
		\raggedright
		\begin{tabular}{ l  p{3.0cm} p{5.4cm} } 
			\toprule
			\textbf{Test} & \textbf{Result} & \textbf{Conclusion} \\\midrule
			
			$H_2O$ Solubility & Not Soluble & C3 is a relatively non-polar molecule \\\hline
			
			$NaOH$ Solubility & Soluble & C3 is at least slightly acidic \\\hline
			
			$NaHCO_3$ Solubility & Not Soluble & C3 is only slightly acidic i.e. not a carboxylic acid\\\hline
			
			Beilstein Test & No green flame, Negative & C3 does not have a halide substituent \\\hline
			
			Soot Formation & Produced soot & C3 is likely aromatic or contains an alkene group \\\hline
			
			$H_2CrO_4$ & Produced dark purple fluid, Negative & C3 is not a primary or secondary alcohol nor is it an aldehyde \\\hline
			
			Bromine & No reaction & C3 is unlikely to have an alkene or aldehyde group \\\hline 
			
			2,4-DNPH & No reaction & C3 does not have a carbonyl group \\\hline 
			
			$KMnO_4$ & Becomes clear with brown precipitate, Positive & C3 likely has double bonds or possibly an activated aromatic \\\hline  
			
			Ferrous Hydroxide & No reaction & No nitro group \\\hline
			
			$AgNO_3$ Test & No reaction & C3 does not have a tertiary alkyl halide moiety \\\hline 
			
			$NaI$ Test & No reaction & C3 does not have a primary alkyl halide moiety \\\hline 
			
			C3 Melting Point & 86 C &  \\\hline
			
			Bromophenol Deriv. MP & 105 C &  \\\hline
			
			$\alpha$-Naphthylurethane Deriv. MP & 207 C &  \\\hline
			
		\end{tabular}
	\end{table}
	\pagebreak
	\subsection{Unknown C4}
	\begin{table}[h]
		\raggedright
		\begin{tabular}{ l  p{3.0cm} p{5.4cm} } 
			\toprule
			\textbf{Test} & \textbf{Result} & \textbf{Conclusion} \\\midrule
			
			$H_2O$ Solubility & Not Soluble & C4 is a relatively non-polar molecule \\\hline
			
			$NaOH$ Solubility & Not Soluble & C4 is not acidic \\\hline
			
			$HCl$ Solubility & Soluble & C4 is slightly basic, almost certainly an amine \\\hline
			
			Beilstein Test & Green flame, Positive & C4 does has a halide substituent \\\hline
			
			Soot Formation & Produced soot & C4 is likely aromatic or contains an alkene group \\\hline
			
			$H_2CrO_4$ & Produced dark purple fluid, Negative & C4 is not a primary or secondary alcohol nor is it an aldehyde \\\hline
			
			Bromine & Cleared, Positive & C4 is likely to have an alkene or aldehyde group, possibly an activated aromatic \\\hline 
			
			2,4-DNPH & No reaction & C4 does not have a carbonyl group \\\hline 
			
			$KMnO_4$ & Becomes darker, Ambiguous & C4 is unlikely to have an alkene group \\\hline  
			
			Ferrous Hydroxide & No reaction & No nitro group \\\hline
			
			$AgNO_3$ Test & No reaction & C4 does not have a tertiary alkyl halide moiety \\\hline 
			
			$NaI$ Test & No reaction & C4 does not have a primary alkyl halide moiety \\\hline 
			
			$Fe(OH)_2$ Test & No reaction & C4 does not have a primary alkyl halide moiety \\\hline
			
			C4 Melting Point & 58 C &  \\\hline
			
			Acetamide Deriv. MP & 167 C &  \\\hline
			
			Benzamide Deriv. MP & 203 C &  \\\hline
			
		\end{tabular}
	\end{table}
	\pagebreak
	\section{Mechanisms}
	\subsection{$AgNO_3$ Reaction}
	\subsection{$NaI$ Reaction}
	\subsection{3,5-Dinitrobenzoyl Chloride \& Cyclopentanol}
	\subsection{Phenylisocyanate \& Cyclopentanol}
	\subsection{2,4-DNPH of Pinacolone}
	\subsection{Semicarbazone of Pinacolone}
	\subsection{Bromophenol of $\alpha$-Naphthol}
	\subsection{Phenylurethane of $\alpha$-Naphthol}
	\subsection{Acetamide of 4-chloroaniline}
	\subsection{Benzamide of 4-chloroaniline}
	\subsection{$Br_2$ with Alkenes}
	\subsection{$H_2CrO_4$ with Cyclopentanol}
	\subsection{$KMnO4$ syn dihydroxylation of alkene}
	
	\pagebreak
	\section{Use \& Description}
	Cyclopentanol is a simple 5 carbon cyclic alcohole  that has a pleasant aroma. At room temperature it is a clear liquid. Cyclopentanol is used as a common pharmaceutical and perfume solvent. It also has a use as a flavoring and scent as it has a pleasant aroma. \\
	
	Pinacolone (3,3-dimethyl-2-butanone) is a clear and viscous liquid at room temperature. It has a citrus like smell. It is primarily used as a precursor for fungicides and herbicides. It is also used as a precusor for various drugs as well for vasodilation and for tuberculosis. \\
	
	$\alpha$-Naphthol is a white solid at room temperature. It is an aromatic compound used in the synthesis in a variety of drugs. It is also used in hair dyes and agrochemicals such as sevin. It has a slightly bad earthy odor.\\
	
	4-Chloroaniline is a dark solid at room temperature. It has a smokey odor. It is used in dyes such as vat red 32, Green 10, and aniline black. It has some limited use in urea based herbicides and can be used to stain and color dyes used in carpets and wooden furniture.\\
	
	{\small
	References: 
	
	National Center for Biotechnology Information. PubChem Compound Summary for CID 7298, Cyclopentanol.\\ https://pubchem.ncbi.nlm.nih.gov/compound/Cyclopentanol. Accessed May 6, 2024.
	

	National Center for Biotechnology Information. PubChem Compound Summary for CID 6416, Pinacolone.\\ https://pubchem.ncbi.nlm.nih.gov/compound/Pinacolone. Accessed May 6, 2024.
	
	ChemicalBook. CAS DataBase List Pinacolone.\\ https://www.chemicalbook.com/ChemicalProductProperty\_EN\_CB5245383.htm. Accessed May 6, 2024.
	
	National Center for Biotechnology Information. PubChem Compound Summary for CID 7005, 1-Naphthol.\\ https://pubchem.ncbi.nlm.nih.gov/compound/1-Naphthol. Accessed May 6, 2024.
	
	National Center for Biotechnology Information. PubChem Compound Summary for CID 7812, 4-Chloroaniline.\\ https://pubchem.ncbi.nlm.nih.gov/compound/4-Chloroaniline. Accessed May 6, 2024.}
	\pagebreak
	\section{Spectroscopic Analysis of Unknowns}
	
	\subsection{C1 - Cyclopentanol}
	The identity of Cyclopentanol was never under dispute. It was the only chemical out of the list of possibilities with derivatives remotely as high as the results. This meant it had nothing to eliminate. Looking at its C-13 NMR there are 3 types of carbon: 2 "normal" and one with an electronegative attachment. This perfectly matches the structure of Cyclopentanol. The other dead giveaway was the presence of only 3 singlet peaks on the H NMR. A cyclic molecule has Hydrogens locked above and below the ring. These smear into singlets something we wouldn't observe with a straight chain molecule. Even if there were close competitors based only on the boiling point of the unknown itself it was only pentanol which wouldn't match this description at all. The mass spectrometer shows a peak at 86 g/mol which matches the mass of cyclopentanol suggesting some of the unbroken molecule makes it all the way through the process.
	
	\subsection{C2 - Pinacolone}
	After cross referencing the list of possible compounds the prime choices were pinacolone, 3-pentanone, and 2-pentanone. 3-pentanone has a major plane of symmetry which would lead to only 3 carbons in the C-13 NMR. 2-pentanone has no symmetries and therefore should have 5 carbons. However, the C-13 NMR shows 4 carbons which lines up with pinacolone which has a tertbutyl group. All peaks in the H NMR are singlets which would not be the case for 3-pentanone as the Hydrogens in 3-pentanone would all have neighbors and experience splitting. The mass spectrometer further supports this conclusion as the highest peak is 100 g/mol which matches pinacolone very well.
	
	\subsection{C3 - $\alpha$-Naphthol}
	For C3 there were 3 close candidates via melting point analysis: 2-Hydroxyphenol, $\alpha$-Naphthol, or 2,5-dimethylphenol. The C-13 NMR shows 10 aromatic carbons which makes it clear that $\alpha$-Naphthol is the correct molecule. The mass spectrometer shows a mass of 144 g/mol which is far higher than the mass of 2-Hydroxyphenol or 2,5-dimethylphenol and matches the mass of $\alpha$-Naphthol. Due to the aromaticity of the molecules it is difficult to read the H NMR in fine detail, but there are clearly more than the 2 aromatic peaks one would expect of 2-hydroxyphenol. Furthermore the lack of non-aromatic Hydrogens also rules out 2,5-dimethylphenol. This makes it clear C3 is $alpha$-Naphthol.
	
	\subsection{C4 - 4-Chloroaniline}
	The positive Beilstein's narrowed most of the choices down to 4-bromoaniline, 4-chloroaniline, and 2,5-dichloroaniline. The H NMR shows 2 doublets between 6 and 8 ppm. This is the aromatic region and suggests that there are 2 sets of hydrogens with one neighboring hydrogen. This eliminates 2,5-dichloroaniline. C 13 NMR concurs with this result showing 4 aromatic carbons. In order to differentiate Chlorine from Bromine we must consider the mass spectrometry. Chlorine shows a roughly 3:1 skew between 35-Cl and 37-Cl. Bromine on the other hand shows a roughly 1:1 skew between 79-Br and 81-Br. The mass spectrometer has a prominent double peak at 127 and 129 g/mol with a roughly 3:1 skew suggesting 4-chloroaniline is the correct compound.
	\pagebreak
	\section{Discussion}
	\subsection{Cyclopentanol}
	For C1 the lack of solubility in water, NaOH, and HCl made it clear it was a neutral molecule. This left only a ketone, aldehyde, or ester as a possible primary functional group. The lack of green flames suggests that there were no halide substituents. The lack of a positive ferrous hydroxide test suggests there's no nitro group. There are no peaks around 2300 $cm^{-1}$ on the IR spectrum for a cyano group. There were also no double peaks near 1000 or 1300 $cm^{-1}$ to suggest an alkoxy group. There was a broad OH peak on the spectrum near about 3300 $cm^{-1}$ suggesting an alcohol. This is backed up by a positive Chromic Acid test which yielded a blue/green precipitate. This suggests it is specifically either a primary or secondary alcohol. 2,4-DNPH does not react with it suggesting it isn't a ketone. Nor is there a C=O bond in the IR Spectra. There are no aromatic C-H bonds on the IR spectra which agrees with the soot test in this case. Neither $Br_2$ or $KmnO_4$ reacted suggesting that there are no double bonds. This suggests that we likely have a simple primary or secondary alcohol of some kind. After producing the 3,5-dinitrobenzoate and phenylurethane derivatives and looking through the chart the only compound that came close was cyclopentanol. Pentanol has a similar boiling point, but no derivatives of compounds with similar boiling points to the original compound have a similar derivative melting point. After NMR analysis this was easily validated.
	
	\subsection{Pinacolone}
	For C2 the lack of solubility in water, NaOH, and HCl made it clear it was a neutral molecule similar to C1. Notably this time Chromic Acid failed to react which means if there is an alcohol it must be a tertiary alcohol, and that it can't be an aldehyde either. 2,4-DNPH did react leaving only a ketone as a possible option. The IR spectra shows a peak near 1750 $cm^{-1}$ which concurs with the qualitative tests. As for secondary groups there were no green flames or soot so a halide or alkene are unlikely. Indeed $Br_2$ and $KMnO_4$ did not react either. Ferrous oxide failed to react as well suggesting no nitro group. The IR spectra suggests no cyano group. This means that our molecule is likely a simple ketone. The melting points of 2,4-DNPH and Semicarbazide derivatives turned out to not match any in the chart particularly well likely due to a lack of purity. However, the original substance had a boiling point of 105 C which almost exactly matches pinacolone, 3-pentanone, and 2-pentanone. The other available substances are too far off to be particularly reasonable. After spectroscopic analysis it becomes clear that pinacolone is the identity of C2.
	
	\subsection{$\alpha$-Naphthol}
	C3 was soluble in NaOH, but not sodium bicarbonate solution suggesting a weakly acidic group. This means it is likely a phenolic compound of some kind. It forms soot upon burning which further backs up that conclusion. The flame does not turn green so it is unlikely to have a halide group. $KMnO_4$ yields a brown precipitate which could mean it has an alkene or could just be because the aromatic ring is able to be oxidatively cleaved or the groups on the aromatic ring could be syn dihydroxylated. It had no reaction with bromine which suggests that there is no alkene substituent. Curiously it will later form a bromophenol derivative which may suggest that this test was a false negative, or it may have been due to a lack of bromine to spur the reaction. It did not react with ferrous oxide suggesting no nitro group nor did the IR spectra suggest there was a cyano group. The IR spectra has aromatic C-H peaks to the left of 3000 $cm^{-1}$ as well as an O-H stretch near 3300 $cm^{-1}$. The unknowns melting point isn't particularly close to any compound in the chart. The closest are 2,5-dimethylphenol, 1-Naphthol ($\alpha$-Naphthol), and 2-hydroxyphenol. However, using the bromophenol derivative would certainly put 1-Naphthol as the most likely candidate. Spectroscopic analysis agrees.
	
	\subsection{4-Chloroaniline}
	C4 was soluble in HCl suggesting an amine as the primary group. It showed both soot and a green flame suggesting it is aromatic and has a halide substituent. Bromine reacted which is strange as this could mean an alkene is attached somewhere on the ring. $KMnO_4$ does not seem to react in the usual manner which contests this. Furthermore there was The bromine may be reacting with the ring in some way due to having an amino group as an activator. Ferrous hydroxide had no reaction suggesting no nitro group. The IR spectra shows the double peak near 3300-3400 $cm^{-1}$ of a primary amine, although it does cover the aromatic C-H stretch. There are no peaks suggesting any further oxygens or nitrile groups. The derivatives along with the presence of a halide leave only 4-chloroaniline, 4-bromoaniline, or 2,5-dichloroaniline as options. Unfortunately only mass spectrometry can reveal which halide it may be. Spectroscopy shows that the chlorine isotope pattern of a 3:1 skew is present and that C4 is chloroaniline.
	
\end{document}