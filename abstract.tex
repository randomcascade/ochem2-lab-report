\thispagestyle{plain}
\begin{center}
	\Large
	\textbf{Organic Chemistry 2 Lab Report}
	
	\vspace{0.4cm}
	\large
	Chemical Identification
	
	\vspace{0.4cm}
	\textbf{Cameron Good}
	
	\vspace{0.9cm}
	\textbf{Introduction}
\end{center}
\paragraph{}
The purpose of this laboratory is to identify four chemical unknowns through determination of physical properties of the unknowns and their derivatives, chemical reactivity, and spectroscopic methods. The unknowns have a primary functional group of an Aldehyde, an Amine, a Ketone, an Alcohol, a Carboxylic Acid, an Ester, or a Phenol group. They may have also have an additional secondary or tertiary functional group of a halogen, a nitro, a cyano, an alkoxy, an alkene, an alkyne, or an aromatic group.

Chemical reactivity can be used to identify substituents on a carbon chain. The ability of a substance to dissolve can used to identify whether a molecule is polar or non-polar or if it is an acid or base. Numerous functional groups can be detected using with qualitative tests. Physical properties like melting points and boiling points can be used to look up chemicals in a reference book that may be potential matches. Finally NMR, IR, and mass spectrometry can be used to estimate the structure and mass of the molecule with a high degree of accuracy.

